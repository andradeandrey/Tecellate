\section{Simulation Rules}
\label{rules}

The simulation is turn-based and runs over a fixed number of turns. The length of the turn is designed to simulate the time to send M packets. At the beginning of each turn, agents receive information about the terrain and other nearby agents. In response, agents can either attempt to move or remain still. During the turn the agents can also listen for messages and choose to broadcast their won. The broadcasts will be recieved by nearby agents during the next turn. When all agents have decided on an action for a turn, the actions are executed and a new turn begins.

\subsection{World and Movement}

The simulation world consists of a $w$*$h$ grid of signed integers. If the number is non-negative, it is the terrain height. Otherwise, the cell is impassable. An agent can move at most one cell up, down, left, or right per turn. The move action takes one extra turn for every absolute height unit difference above 1. If more than one agent attempt to move into the same cell, they are destroyed.

\subsection{Messages}

The broadcast messages agents can send are a simpilification of radio communication. There are N frequencies an agent can broadcast on or listen too. An agent may broadcast a message of 1024 bytes on a single frequency. The agent may also listen for messages. Each agent can listen on M frequencies each turn where M is strictly less than N.

Like radio communication, the messages broadcast decay over distance. When an agent listens to a frequency the agent always recieves data. However, if no one is broadcasting on that frequency the agent will recieve random data back. Since the messages decays over distance, even if an agent hears it some parts of the message may be corrupted. The farther away the more corruption is in the message.

Finally, if messages cross each other the following algorithm will be used to resolve the conflict. First, for each message the message will be corrupted based on its distance to its source, just as in the single broadcaster senario. Then, for each bit in each message their will be a probability K that the bit will be heard. The probablity will also be based on the distance to the source of the message. All of the bits that are heard are XORed together creating a composite message. If not bit are heard for a particular bit location a random value is chosen for that bit.

\subsection{Perception}

Agents are informed of their global position in the world. They are also informed of the position of any agents within a maximum number of cells.

\subsection{Food}

Agents must search the grid to find food. If their coordinator calculates that their food counter is below zero, they disappear from the grid.

\section{Usefulness of the Simulation}

The simulation provides a close enough approximation of the real world to allow testing of high-level coordination algorithms to be tested using a very simple API.

The rules for movement (1 delayed turn per cell height difference) and perception provide an opportunity to experiment with pathfinding algorithms in an environment where little information outside a limited area is available, similar to the conditions of the DARPA autonomous vehicle events.

The rules for communication force agents to communicate with distant agents by passing messages through nearby agents, creating ad-hoc wireless networks. Our messaging system is meant to model the difficulties of real world radio communication. This will make our simulation a useful tool for simulating adhoc wireless networks allowing protocol research to be conducted without hardware.
