\section{Simulation Rules}
\label{rules}

The simulation is turn-based and runs over a fixed number of turns. At the beginning of each turn, agents receive information about the terrain and other nearby agents. In response, agents can either attempt to move or remain still, as well as simultaneously broadcast a message. When all agents have decided on an action for a turn, the actions are executed and a new turn begins.

\subsection{World and Movement}

The simulation world consists of a $w$x$h$ grid of signed integers. If the number is non-negative, it is the terrain height. Otherwise, the cell is impassable. An agent can move at most one cell up, down, left, or right per turn. The move action takes one extra turn for every absolute height unit difference above 1.

If more than one agent attempt to move into the same cell, they are destroyed.

\subsection{Messages}

An agent may broadcast a message of 1024 bytes or less every turn regardless of any other action it takes. The broadcast system is a simplification of radio communication: agents will receive other agents' broadcasts, but noise will be introduced proportional to the square of the distance between the agents.

\subsection{Perception}

Agents are informed of their global position in the world. They are also informed of the position of any agents within a maximum number of cells.

\section{Usefulness of the Simulation}

The simulation provides a close enough approximation of the real world to allow testing of high-level coordination algorithms to be tested using a very simple API.

The rules for movement (1 delayed turn per cell height difference) and perception provide an opportunity to experiment with pathfinding algorithms in an environment where little information outside a limited area is available, similar to the conditions of the DARPA autonomous vehicle events.

The rules for communication force agents to communicate with distant agents by passing messages through nearby agents, creating ad-hoc wireless networks.
