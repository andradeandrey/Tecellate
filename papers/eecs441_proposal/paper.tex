\documentclass[12pt]{article}
\usepackage[margin=1in]{geometry}
\usepackage{graphicx}
\usepackage{setspace}
\usepackage{url}

\title{Tecellate: A Distributed Environment for Ad Hoc Wireless Network Simulations}
\author{
        Steve Johnson (srj15@case.edu)\\
        Case Western Reserve University\\
        \and Tim Henderson (tah35@case.edu)\\
        Case Western Reserve University
}
\date{\today}

\begin{document}
\doublespacing
\maketitle

% \begin{multicols}{2}

\begin{abstract}
    We want to build a simulation of a world in which autonomous agents can communicate using semi-realistic radio signals. The simulation itself can be distributed over multiple machines to allow for very large simulations. The purpose will be to test wireless ad-hoc network routing algorithms under arbitrarily harsh conditions.
    This project will be based on work done in the fall of 2010 in EECS 423 (Distributed Systems). The current implementation allows for very rudimentary setup and movement but not much else.
\end{abstract}

\section{Introduction}

Network simulations have become important tools for researching mobile networks. Most research groups do not have the ability to purchase and use thousands of mobile devices for research and have thus turned to simulation. However, as many previous simulation authors have noted, simulations will always sacrifice some element of realism. The authors have decided to take a novel approach in these sacrifices adding a new way to test wireless mobile networks.

The simulation we propose to develop would operate at the link layer. That is there is every ``agent,'' the actors in the simulation, have a radio that sends and receives bytes. Thus, unlike the real world where the radio would have to convert the radio waves to a digital signal, in the simulation this is assumed to be done for you. However, it is up the agents to ensure that their link level connectivity is accurate. The simulation includes complications like interference from other sources and exponential drop-off of power as agents move away from the source of a transmission.

In the physical world, devices move around with purpose and transmit with purpose. It is the authors assertion that in simulation the same should be true. However, modeling the purpose of actual people is not feasible. Instead, the authors propose to give the agents an alternative purpose: staying alive. Each agent has to ``eat'' to survive. Each turn it ``eats'' a unit of food. If it runs of food it dies. The goal of the agents is to stay alive as long as possible. Thus, it must find sources of food within the simulation. This added, game like element provides the agents with goals and a purpose for communication.

By modeling the communication at the link layer and adding game like elements to our simulation we hope to provide an interesting test bed for networking and AI algorithms. The game like element in our simulation will help ensure the agents move around purposefully in the simulation grid. The link layer simulation ensures the communication problems which must be overcome for reliable communication is similar to real world problems. While, we do not think simulation approach would be the final word in networks simulation it should provide a useful platform for experimentation.

\section{Literature Survey}

There are a few existing simulations of wireless ad hoc networks that operate above the link layer. One of these is ns2, which has been around for several years \cite{ns2}. ns2 simulates multiple kinds of networks and protocols, and it has been popular among researchers. However, it is difficult to scale above a few hundred nodes \cite{swans}. It has only been documented to scale to a few thousand nodes.

GloMoSim is a newer simulator \cite{glomosim}. It serves a similar purpose to ns2 and can simulate both wired and wireless networks, but it is more scalable in that it can run its event loop in parallel \cite{swans}. It can scale to tens of thousands of nodes.

SWANS is a simulator created purely for ad hoc wireless network research in response to the inadequacy of ns2 and GloMoSim for ad hoc wireless network research \cite{swans}. It has a different configuration/modeling style from other solutions which allows it to accurately simulate millions of nodes on 2004 hardware.

A specialized wireless ad hoc network simulator is $madhoc$, a system that attempts to accurately model wireless network characteristics in metropolitan settings \cite{madhoc}. The authors argue that other simulators' node movement algorithms do not accurately model real world environments, and consequently spend a great deal of effort modeling the characteristics of metropolitan environments.

These simulators differ from Tecellate in that they simulate a higher level of the network and provide no inherent goal for communicating agents. They also simulate radio signal propagation much more accurately, which affects the computation requirements drastically. They have sophisticated physics-based models of signal strength which must be updated and accounted for when sending and receiving transmissions, while we propose to use simpler probabilistic methods at the bit level.


\section{Simulation Rules}
\label{rules}

The simulation is turn-based and runs over a fixed number of turns. At the beginning of each turn, agents receive information about the terrain and other nearby agents. In response, agents can either attempt to move or remain still, as well as simultaneously broadcast a message. When all agents have decided on an action for a turn, the actions are executed and a new turn begins.

\subsection{World and Movement}

The simulation world consists of a $w$x$h$ grid of signed integers. If the number is non-negative, it is the terrain height. Otherwise, the cell is impassable. An agent can move at most one cell up, down, left, or right per turn. The move action takes one extra turn for every absolute height unit difference above 1.

If more than one agent attempt to move into the same cell, they are destroyed.

\subsection{Messages}

An agent may broadcast a message of 1024 bytes or less every turn regardless of any other action it takes. The broadcast system is a simplification of radio communication: nearby agents within a maximum radius will receive the message. Other agents will not hear it.

\subsection{Perception}

Agents are informed of their global position in the world. They are also informed of the position of any agents within a maximum number of cells.

\subsection{Food}

Agents must search the grid to find food. If their coordinator calculates that their food counter is below zero, they disappear from the grid. This rule would give agent programmers incentives to introduce distributed communication networks to inform faraway agents of abundant new food sources.

\section{Usefulness of the Simulation}

The simulation provides a close enough approximation of the real world to allow testing of high-level coordination algorithms to be tested using a very simple API.

The rules for movement (1 delayed turn per cell height difference) and perception provide an opportunity to experiment with pathfinding algorithms in an environment where little information outside a limited area is available, similar to the conditions of the DARPA autonomous vehicle events.

The rules for communication force agents to communicate with distant agents by passing messages through nearby agents, creating ad-hoc wireless networks.


\section{Current State of the Project}

The simulation contains \textbf{agents} which must be sent stimuli to respond to. The current implementation consists of a \textbf{master process} which connects to several \textbf{coordinator processes} that are responsible for keeping the state of disjoint sets of agent processes.

The master process is responsible for connecting to the coordinators, sending them configurations, and waiting for their final responses. The coordinator connects to neighbor coordinators and agents, serves requests for information about the state of its agents in the last turn, and processes its own agents by requesting information from its neighbors and from its agents. In this way, the coordinators operate in lock-step, turn by turn, until all turns have been processed by all coordinators.

The current version of the simulation only supports agent movement and agent death. Agents are killed when they both occupy the same cell. The master process and coordinators do not attempt to balance load after the simulation has begun.


\section{Distributing Work}

Simulating thousands of agents and millions of grid cells simultaneously at a reasonable speed will require multiple machines. One server, the \emph{game coordinator}, will be responsible for starting the game and reporting the rules. The other $n_ac$ servers will be \emph{agent coordinators}, responsible for delivering messages, taking and executing agent orders, and reporting success or failure.

Agents interact in two primary ways: messages and movement conflict/support. Both are only effective within a certain proximity. These properties suggest that agents should be assigned to coordinators based on proximity. There are multiple ways to perform this assignment. This paper will consider three strategies: static assignment, fixed region distribution, and dynamic grid distribution.

\subsection{Static Assignment}

The simplest way to assign agents to coordinators is to assign $\frac{n_agents}{n_ac}$ agents to each agent coordinator at random or based on some positioning heuristic. No agents will ever change their coordinator, but there may be significant overhead from close agents on different coordinators attempting to move and communicate. Also, each agent coordinator will have to be in constant communication with all other coordinators because it cannot know which other coordinators contain agents close to its own.

\subsection{Fixed Region Distribution}

One way to alleviate the communication problem is to assign each agent coordinator a fixed portion of the grid to be responsible for. There will never be a recalculation of bounds and each coordinator has a constant set of neighbor coordinators to communicate with. Each turn, adjacent coordinators will exchange messages, and order confirmations for agents near their shared border, and full agent information transfers for agents moving from one coordinator's region to a neighbor's region.

This strategy requires the least communication between agent coordinators when agents move relatively little, but requires much communication when many agents are near region borders or move between borders often. It also introduces inefficiency in work distribution when disproportionate numbers of agents are gathered in small regions.

\subsection{Dynamic Grid Distribution}

Like fixed region distribution, dynamic grid distribution assigns each agent coordinator a region of the grid to be responsible for. However, the regions themselves are arranged in a meta-grid. This grid arragenement allows rows and columns to be resized while the simulation is running, partially alleviating the work distribution problem.

\subsection{Strategy for First Version}

The initial implementation of Tecellate will use static assignment to test communication protocols and basic mechanics. Then a fixed region distribution over a meta-grid will be introduced. If there is time, dynamic grid distribution will be implemented.

\section{Conclusion}

\nocite{*}
\bibliographystyle{acm}
\bibliography{bibliography}

\end{document}
